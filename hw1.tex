\documentclass [a4paper,12pt]{article}
\usepackage{amsmath,amsthm,amssymb}
\usepackage{mathtext}
\usepackage[T1,T2A]{fontenc}
\usepackage[utf8]{inputenc}
\usepackage[english,russian]{babel}


\title{Домашние задание №1 по дисциплине "Теория случайных процессов"}

\author{Головатских Марк БПМ-16-1}

\begin{document}

\maketitle
\pagenumbering{gobble}
\newpage
\pagenumbering{arabic}
Матрица одношаговых переходов:\\

$P = \left(
\begin{matrix}
\frac{1}{2} & \frac{1}{2} & 0 & 0&0&0&0&0\\
1 & 0 & 0 & 0&0&0&0&0&\\
0 & 0 & 0 & 0&\frac{1}{3}&0&\frac{2}{3}&0\\
0 & \frac{1}{2} & \frac{1}{2} & 0&0&0&0&0\\
0 & 0 & 0 & \frac{3}{4}&0&0&0&\frac{1}{4}\\
0 & 0 & 0 & 0&0&0&0&1\\
0 & 0 & 0 & 0&0&1&0&0\\
0 & 0 & 0 & 0&0&0&1&0\\
\end{matrix}
\right) $\\

Построим граф этой марковской цепи:\\


\end{document}
